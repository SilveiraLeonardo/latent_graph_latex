Interagimos com dados estruturados em forma de grafos em diversos momentos do nosso dia-a-dia. Alguns exemplos são no uso de redes sociais, ao recebermos recomendações de compra na internet e quando pesquisamos em bases de dados de conhecimento.
%
Ao longo da última década, diversos métodos de aprendizado de máquina para grafos foram propostos e tiveram sucesso em suas aplicações, explorando as poderosas características dessa estrutura de dados: A capacidade de discretizar as informações em entidades e relacioná-las entre si.
%
Apesar do sucesso em aplicações com dados estruturados em forma de grafos, a parcela de dados não estruturados, como imagens e texto, onde esses métodos não podem ser aplicados, é muito maior.
%
Neste projeto, pretendemos desenvolver um método que seja capaz de extrair o grafo não observável, ou latente, de dados não estruturados.
%
Isso permitirá explorarmos as características relacionais e composicionais dos dados em um universo muito mais abrangente do que é possível atualmente, indo além dos dados que naturalmente se apresentam estruturados como grafos. 
%
O método proposto irá trabalhar com dados não estruturados onde conhecemos as entidades presentes, mas não suas relações. 
%
O método deve ser capaz de inferir grafos dinâmicos, onde as arestas do grafo podem sofrer alterações durante seu ciclo de vida, e também considerar dependências entre as arestas do grafo durante o processo gerador, como ocorre em grafos reais. 
%
Adicionalmente, gostaríamos de explorar se o conhecimento do grafo latente pode melhorar a performance de dados já estruturados em tarefas de aprendizado de máquina, atuando, por exemplo, como uma forma de \emph{data augmentation}.