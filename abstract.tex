We interact with graph structured data in several moments of our daily lives. Examples of these interactions are in the use of social media, when buying on the internet and receiving product recommendations, and when we search on knowledge bases.
%
In the last decade, many methods for learning from graph structured data were proposed and succeeded in their applications, exploiting the powerful capabilities that graph data has of discretizing the data information in entities and their features, and relating them together.
%
Despite this, the sheer amount of data that is unstructured represents a much larger share than the naturally structured data. This represents a large untapped source of data where the proposed methods cannot be applied.
%
In this research, our goal is to develop a method capable of inferring the latent graph from unstructured data, from which we know the entities, but not their relations.
%
Our method must be capable of inferring dynamic graphs, in which the relations of the nodes may change during the graph life-cycle, and it also must consider that  connections between the elements of the graph are dependent on one another during the generation phase, as it is the case in real world graphs.
%
Additionally, we will study the performance gain by using the inferred latent graph in already structured data, in place of the \emph{natural} graph, or combined with it.
